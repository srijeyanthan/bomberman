%%%%%%%%%%%%%%%%%%%%%%%%%%%%%%%%%%%%%%%%%
% Beamer Presentation
% LaTeX Template
% Version 1.0 (10/11/12)
%
% This template has been downloaded from:
% http://www.LaTeXTemplates.com
%
% License:
% CC BY-NC-SA 3.0 (http://creativecommons.org/licenses/by-nc-sa/3.0/)
%
%%%%%%%%%%%%%%%%%%%%%%%%%%%%%%%%%%%%%%%%%

%----------------------------------------------------------------------------------------
%	PACKAGES AND THEMES
%----------------------------------------------------------------------------------------

\documentclass{beamer}

\mode<presentation> {

% The Beamer class comes with a number of default slide themes
% which change the colors and layouts of slides. Below this is a list
% of all the themes, uncomment each in turn to see what they look like.

%\usetheme{default}
%\usetheme{AnnArbor}
%\usetheme{Antibes}
%\usetheme{Bergen}
%\usetheme{Berkeley}
%\usetheme{Berlin}
%\usetheme{Boadilla}
%\usetheme{CambridgeUS}
%\usetheme{Copenhagen}
%\usetheme{Darmstadt}
%\usetheme{Dresden}
%\usetheme{Frankfurt}
%\usetheme{Goettingen}
%\usetheme{Hannover}
%\usetheme{Ilmenau}
%\usetheme{JuanLesPins}
%\usetheme{Luebeck}
\usetheme{Madrid}
%\usetheme{Malmoe}
%\usetheme{Marburg}
%\usetheme{Montpellier}
%\usetheme{PaloAlto}
%\usetheme{Pittsburgh}
%\usetheme{Rochester}
%\usetheme{Singapore}
%\usetheme{Szeged}
%\usetheme{Warsaw}

% As well as themes, the Beamer class has a number of color themes
% for any slide theme. Uncomment each of these in turn to see how it
% changes the colors of your current slide theme.

%\usecolortheme{albatross}
%\usecolortheme{beaver}
%\usecolortheme{beetle}
%\usecolortheme{crane}
%\usecolortheme{dolphin}
%\usecolortheme{dove}
%\usecolortheme{fly}
%\usecolortheme{lily}
%\usecolortheme{orchid}
%\usecolortheme{rose}
%\usecolortheme{seagull}
%\usecolortheme{seahorse}
%\usecolortheme{whale}
%\usecolortheme{wolverine}

%\setbeamertemplate{footline} % To remove the footer line in all slides uncomment this line
%\setbeamertemplate{footline}[page number] % To replace the footer line in all slides with a simple slide count uncomment this line

%\setbeamertemplate{navigation symbols}{} % To remove the navigation symbols from the bottom of all slides uncomment this line
}

\usepackage{graphicx} % Allows including images
\usepackage{booktabs} % Allows the use of \toprule, \midrule and \bottomrule in tables
\usepackage{tikz} % For PGFs
\usepackage{hyperref} % For links
\usepackage{mathtools} % for Equations
\usepackage{lmodern}
\usepackage{amsmath}

\AtBeginSection[]{%
  \begin{frame}<beamer>
    \frametitle{Outline}
    \tableofcontents[
    currentsubsection,
    hideothersubsections,
    sectionstyle=show/hide, 
    subsectionstyle=show/shaded, 
    ]
  \end{frame}
  \addtocounter{framenumber}{-1}% If you don't want them to affect the slide number
}

%----------------------------------------------------------------------------------------
%	TITLE PAGE
%----------------------------------------------------------------------------------------

\title[SSDS]{An Architecture for a Secure Service Discovery Service} % The short title appears at the bottom of every slide, the full title is only on the title page

\author{Steven E. Czerwinski, Ben Y. Zhao, Todd D. Hodes, Anthony D. Joseph, Randy H. Katz} % Your name
\institute[UC Berkeley] % Your institution as it will appear on the bottom of every slide, may be shorthand to save space
{
University of California, Berkeley \\ % Your institution for the title page
\medskip
\textit{\{czerwin, ravenben, hodes, adj, randy\}@cs.berkeley.edu} % Your email address
}
\date{\today} % Date, can be changed to a custom date

\begin{document}

\begin{frame}
\titlepage % Print the title page as the first slide
\end{frame}

\begin{frame}
\frametitle{Overview} % Table of contents slide, comment this block out to remove it
\tableofcontents % Throughout your presentation, if you choose to use \section{} and \subsection{} commands, these will automatically be printed on this slide as an overview of your presentation
\end{frame}

%----------------------------------------------------------------------------------------
%	PRESENTATION SLIDES
%----------------------------------------------------------------------------------------

%------------------------------------------------
\section{Introduction} % Sections can be created in order to organize your presentation into discrete blocks, all sections and subsections are automatically printed in the table of contents as an overview of the talk
%------------------------------------------------

\subsection{Introduction} % A subsection can be created just before a set of slides with a common theme to further break down your presentation into chunks

\begin{frame}
\frametitle{Introduction}
\begin{itemize} 
\item Large scale deployment of networks and devices
\item Challenge: locate a service for a task out of thousands
\item Secure and trusted services with minimum client intervention
\item Ninja SDS: scalable, fault-tolerant and secure repository
\item Repository: description of services and existing, running, services
\item Hierarchical load-balancing and application level query routing
\end{itemize}
\end{frame}

%------------------------------------------------
\section{Design Concepts} 
\subsection{Announce/Listen model} 

\begin{frame}
\frametitle{Announcement-based Information Dissemination}
\begin{itemize}
\item \alert<+>{Failure does not require a separate service} \hfill \\
It's sufficient to listen to the periodic multicast announcements to update the cache/database
\item \alert<+>{Bootstrapping} \hfill \\
Clients discover an SDS server by listening to a multicast address. Client can solicit asynchronous announcements
\item \alert<+>{Eventual Consistency} \hfill \\
Eventual consistency vs. transactional semantic
\end{itemize}
\end{frame}

%------------------------------------------------
\subsection{Hierarchical Organisation}
\begin{frame}
\frametitle{Hierarchical Organisation}
\begin{block}{SDS server hierarchy}
\begin{itemize}
\item under heavy load, a child is spawned - shares load with parent
\item the domain of an SDS server (fractional subnet, etc) is the network extent it covers
\end{itemize}
\end{block}
\end{frame}

%------------------------------------------------
\subsection{XML Service Descriptions}
%------------------------------------------------
\begin{frame}
\frametitle{XML Service Descriptions}
\begin{block}{XML Service Descriptions}
\begin{enumerate}
\item More flexible than <key,value> pairs
\item Service description validation against a set schema
\item Database schema vs. DTD: flexibility, backward-compatibility
\end{enumerate}
\end{block}
\end{frame}

%------------------------------------------------
\subsection{Privacy and Authentication}
%------------------------------------------------

\begin{frame}
\frametitle{Privacy and Authentication}
\begin{block}{Contents}
\begin{itemize}
\item Hybrid symmetric and asymmetric cryptography used 
\item Each component has a public key and a principal name
\end{itemize}
\end{block}
\end{frame}

%------------------------------------------------
\section{Architecture}
\begin{frame}
\frametitle{SDS Server}
\begin{itemize}
\item Global multicast channel to send authenticated messages
\item Authenticated advertisements contain:
\begin{itemize}
\item Certificate Authority and Capabilities Manager contact
\item Address for sending service announcements
\item Service annoucement rate
\end{itemize}
\item Aggregate rate set by administrators
\item Should several servers fail, recovery is cascading from the top
\item Privacy and authentication assured by the {\it secure one-way service broadcast}
\end{itemize}
\end{frame}

%------------------------------------------------

\begin{frame} % Need to use the fragile option when verbatim is used in the slide
\frametitle{Services}
\begin{enumerate}
\item \alert<+>{Continously listen on the global multicast channel for SDS server announcements}
\item \alert<+>{Multicast its service descriptions to the proper channel/frequency}
\item \alert<+>{Set appropriate capabilities by contacting the Capabilities Manager}
\end{enumerate}
\end{frame}

%------------------------------------------------

\begin{frame} % Need to use the fragile option when verbatim is used in the slide
\frametitle{Capabilities Manager}
\begin{enumerate}
\item \alert<+>{Contacted by services}
\item \alert<+>{Specifies an access control list}
\item \alert<+>{Generates and stores appropriate capabilities}
\end{enumerate}
\end{frame}

%------------------------------------------------
\begin{frame}
\stepcounter{beamerpauses}
\frametitle{Secure Communications}
\begin{itemize}[<+->]
\item Authenticated Server Annoucements 
\item Secure One-Way Service Description Annoucements
\item Authenticated RMI
\end{itemize}
\end{frame}

%------------------------------------------------
\section{Wide Area Support}
\begin{frame}
\stepcounter{beamerpauses}
\frametitle{Wide Area Support}
\begin{enumerate}[<+->]
\item Lorem Ipsum   
\item Lorem Ipsum
\item Lorem Ipsum
\end{enumerate}
\end{frame}
%------------------------------------------------

\subsection{Adaptive Server Hierrachy Management}
\begin{frame}[fragile] % Need to use the fragile option when verbatim is used in the slide
\frametitle{Alternatives}
\end{frame}

%------------------------------------------------

\begin{frame}
\Huge{\centerline{Questions ?}}
\end{frame}

%----------------------------------------------------------------------------------------

\end{document}
